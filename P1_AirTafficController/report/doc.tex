%%%%%%%%%%%%%%%%%%%%%%%%%%%%%%%%%%%%%%%%%%%%%%%%%%%%%%%%%%%%%%%
% AirTafficController
% David Miguel Lozano
% Javier Martínez Riberas
% Universidad de Burgos - Diciembre 2016
%%%%%%%%%%%%%%%%%%%%%%%%%%%%%%%%%%%%%%%%%%%%%%%%%%%%%%%%%%%%%%%

% -------------------------------------------------------------
% Preamble
% -------------------------------------------------------------
\documentclass[a4paper,12pt,titlepage]{article}

% -------------------------------------------------------------
% Packages
% -------------------------------------------------------------
\usepackage[utf8]{inputenc}	% Unicode support
\usepackage[T1]{fontenc}		% Font encoding
\usepackage[spanish]{babel}	% Languaje
\usepackage{lmodern}	% Typeface 
\usepackage{textcomp} % Special symbols
\usepackage{graphicx}	% Add pictures
\usepackage{pgfplots} % Graphs and charts
\usepackage{hyperref}	% Add a link to index entries
\usepackage{amsmath}	% Advanced math typesetting
\usepackage{amsfonts}	% Mathematical formulas
\usepackage{amssymb}	% Extended symbol collection
\usepackage{listings}	% Code formatting and highlighting
\usepackage{xcolor}		% Color package
\usepackage{enumitem}	% Customizing lists
\usepackage{parskip}	% Paragraph styles
\usepackage[a4paper]{geometry} 		% Margins
\usepackage[numbers,sort]{natbib}	% Bibliography management
\usepackage{booktabs}							% Tables

% -------------------------------------------------------------
% Configuration
% -------------------------------------------------------------
% Images path
\graphicspath{ {img/} }
% Graphs configuration 
\pgfplotsset{width=\textwidth,compat=1.9}
% Hyperlinks coloring
\hypersetup{
	colorlinks,
	linkcolor={green!40!black},
	citecolor={blue!50!black},
	urlcolor={blue!80!black}
}
% Define HRule
\newcommand{\HRule}[1]{\rule{\linewidth}{#1}}
% Define listings styles
\definecolor{codebg}{HTML}{EEEEEE}
\definecolor{codeframe}{HTML}{CCCCCC}
\definecolor{comments}{HTML}{009900}
\lstset{
  language=Matlab, 								% Programming language 
  backgroundcolor=\color{codebg},	% Background color
  frame=single, 									% Add frame around code
	framesep=10pt,									% Padding
	rulecolor=\color{codeframe},		% Don't change frame color
	upquote=true,										%
	breakatwhitespace=true,					% Break line only in spaces
	keepspaces=true,								% Keep indentation
	tabsize=2,											% Tab size
	title=\lstname, 								% Show filename as caption
	basicstyle=\ttfamily, 					% Size and font
  keywordstyle=\color{black}\ttfamily,
	commentstyle=\color{comments},	% Color of comments
  morecomment=[l][\color{magenta}]{\#}
}
% Define style table
\setlength{\heavyrulewidth}{1.5pt}
\setlength{\abovetopsep}{4pt}
\begin{document}
% -------------------------------------------------------------
% Cover
% -------------------------------------------------------------
\author{David Miguel Lozano \ Javier Martínez Riberas}
\title{P5: AirTafficController v2}
\date{07-11-2016}

\begin{titlepage}
	\centering
	\includegraphics[width=0.16\textwidth]{ubu-logo.png}\par
	\vspace{0.3cm}
	{\scshape\LARGE Universidad de Burgos \par}
	\vfill
	{\scshape\Large Computación Neuronal y Evolutiva \par}
	\HRule{2pt}
	{\huge\bfseries P5: AirTafficController v2 \par}
	\HRule{2pt}
	\\ [0.5cm]
	{Analizar y desarrollar un algoritmo genético que realice el cálculo automático de la distribución más conveniente de vuelos que solicitan aterrizar en un determinado aeropuerto sujetos a varias restricciones.}
	\vfill
	Estudiantes:\par
	{\Large\scshape David Miguel Lozano \\ Javier Martínez Riberas \par}
	\vfill
	Profesor de la asignatura:\par
	\textsc{Bruno Baruque Zanón}
	\vfill
	{\large 1º semestre 2016 \par}
\end{titlepage}

% -------------------------------------------------------------
% Contents
% -------------------------------------------------------------
\newpage
\tableofcontents
\begin{appendix}
  %\listoffigures
  %\listoftables
\end{appendix}

% -------------------------------------------------------------
% Body
% -------------------------------------------------------------
\newpage

\section{Introducción}

El objetivo de esta práctica es desarollar un programa que implemente un algoritmo genético que permita al usuario realizar el calculo automático de la distribución
más conveniente de vuelos que solicitan aterrizar en un determinado aeropuerto sujetos a varias restricciones. Se experimentará modelando el problema con restricciones débiles y empleando optimización multi-objetivo. Así mismo, se analizará la conveniencia de la solución propuesta y se reflexionará sobre los resultados obtenidos.

La aplicación tiene que ser capáz de adaptarse a las siguientes condiciones:

\begin{itemize}[noitemsep]
	\item El número de pistas del aeropuerto debe ser un parámetro configurable.
	\item Cada pista acepta un conjunto determinado de tipos de aviones.	
	\item El número de aviones en cada situación puede variar.
	\item Los aviones tienen un programa de vuelo que incluye:
	\begin{itemize}[noitemsep]
		\item ETA (\textit{Estimated Time of Arrival}): tiempo estimado de llegada calculado en el momento del despegue.
		\item Tipo de avión: heavy / big / small. Condiciona el tiempo necesario para su aterrizaje.
	  \item Vuelo asociado (opcional): vuelo que se desea que aterrice con la menor diferencia de tiempo del vuelo actual.
	\end{itemize}
\end{itemize}

El programa tiene que conseguir obtener de forma automática la mejor asignación posible de vuelos a aterrizar en cada pista, de forma que se cumplan los siguientes objetivos:

\begin{itemize}[noitemsep]
	\item El tiempo de espera de los vuelos en su conjunto sea el menor posible.
	\item Solo los tipos de aviones autorizados aterricen en las pistas.
	\item Que el tiempo transcurrido entre aterrizajes de vuelos asociados sea el menor posible. 
\end{itemize}

Para su resolución se hará uso de la librería para Java JCLEC (Java Class Library for Evolutionary Computation) \citep{web:jclec}. La cual, proporciona un framework para programación evolutiva que da soporte, entre otras cosas, a los algoritmos genéticos.

\section{Solución propuesta}

A continuación se detalla la codificación y configuración del problema para ser resuelto con un algoritmo genético.

\subsection{Representación de los individuos}

Para representar los individuos se ha utilizado un array de enteros ordenado (\lstinline|OrderArrayIndividual|). Cada posición del array representaba un avión y el array estaba ordenado por orden de llegada. De tal manera, que la primera posición se correspondía con el primer avión en llegar y el valor de cada posición indicaba el identificador del avión.

Ejemplo de genotipo:

\begin{center}
$
\begin{bmatrix}
	2 & 3 & 1 & 4
\end{bmatrix}
$
\end{center}

Representa que el primer vuelo en aterrizar fue el 2, seguido del 3, 1 y 4.

\subsection{Esquema evolutivo}

Se ha utilizado el algoritmo SGE (\textit{Simple Generational and Elitist}). Se trata de un algoritmo elitista que asegura que, en cualquier momento, sólo los mejores individuos pasen a la siguiente generación \citep{jclec:sge}.

\subsection{Función de fitness}

Para evaluar los individuos, como el genotipo se encontraba ordenado por orden de llegada, se iba iterando sobre él y planificando cada vuelo. La asignación de la pista se realizaba minimizando el ATA, de tal forma, que se asignaba la primera pista que quedase libre. Al finalizar la planificación, se comprobaban las restricciones incumplidas.

Por último, el cálculo del fitness se realizó teniendo en cuenta los siguientes parámetros:

\begin{enumerate}[noitemsep]
	\item Minimizando el retraso acumulado. Es decir, el sumatorio de la diferencia entre el ATA y el mínimo ETA de cada avión.
	\item El número de restricciones de pistas violadas.
	\item Minimiando el tiempo entre aterrizajes de vuelos asociados. Es decir, la diferencia en valor absoluto de los dos ATAs.
\end{enumerate}

En la primera parte de la prática, los dos últimos parámetros se implementaron como restricciones débiles. Dando más importancia a la restricción de pistas.

En la segunda parte de la práctica, se ha implementado mediante optimización multiobjetivo, por los que los tres parámetros se han tenido en cuenta por igual.

\subsection{Inicialización}

La población inicial se inicializa de forma aleatoria. Se ha utilizado el generador \lstinline|Ranecu|, un generador lineal congruencial avanzado con un periodo aproximado de $10^{18}$ \citep{jclec:ranecu}.

\subsection{Criterio de parada}

El criterio de parada se ha establecido en 1.000 generaciones por defecto.

\subsection{Criterio de selección}

Para seleccionar un subconjunto de la población se han analizado los siguientes algoritmos:

\begin{enumerate}[noitemsep]
	\item \lstinline|RouletteSelector|: selección por ruleta \citep{jclec:roulette}.
	\item \lstinline|RandomSelector|: selección aleatoria \citep{jclec:random}.
\end{enumerate}

\subsection{Operador de cruce}

Para obtener un nuevo individuo basado en el genotipo de sus padres se han analizado los siguientes algoritmos:

\begin{enumerate}[noitemsep]
	\item \lstinline|OrderOXCrossover|: OX Crossover. 
	\item \lstinline|OrderPMXCrossover|: PMX Crossover.
\end{enumerate}

La probabilidad de cruce se estableció en un 75\% por defecto.

\subsection{Operador de mutación}

Cada gen del genotipo de un individuo tiene, por defecto, un 3\% de probabilidad de mutar. Se han analizado los siguientes algoritmos de mutación:

\begin{enumerate}[noitemsep]
	\item \lstinline|Order2OptMutator|: mutación 2-opt del genotipo.
	\item \lstinline|OrderSublistMutator|: mutación de una sublista del genotipo aleatoriamente.
\end{enumerate}

\subsection{Criterio de reemplazo}

Se ha utilizado \lstinline|OrderArrayCreator|, mediante el cual, los hijos reemplazan directamente a los padres. Para preservar el elitismo, si la mejor sulución de la generación anterior no sobrevive, la peor solución se reemplaza por una nueva.

No se han analizado más algoritmos de reemplazo ya que la librería sólo proporciona este para trabajar con \lstinline|OrderArrayIndividual|.

\subsection{Implementación}

Para la importación de la librería JCLEC se ha creado una dependencia Maven de esta. Se ha publicado en el siguiente repositorio: \href{https://github.com/davidmigloz/jclec\_maven\_repo}{JCLEC Maven Repository}. 

*Se ha añadido el paquete \lstinline|orderarray| a la versión base de la libería. Ya que, en la versión original sólo se incluye con los ejemplos. Además, se han sustituido las clases con errores por las versiones corregidas proporcionadas con los materiales de la prática.

La aplicación cuenta con las siguientes clases:

\begin{itemize}[noitemsep]
	\item \lstinline|Run|: permite lanzar la aplicación seleccionando por parámetro el archivo de vuelos deseado.
	\item \lstinline|AirTrafficController|: implementación del algoritmo genético.
	\item \lstinline|Airport|: clase que modela un aeropuerto. Posee la lógica para seleccionar la mejor pista para un determinado avión. Además, permite conocer el retraso acumulado, el momento en el que aterrizó el último avión, el total de restricciones de pista violadas y el acumulado del retraso de aviones asociados.
	\item \lstinline|Runway|: clase que modela una pista del aeropuerto. Posee la lógica para calcular cuando estará libre para que aterrice un determinado tipo de avión y conocer cuantas veces se ha violado la restricción de pista.
	\item \lstinline|Flight|: clase que modela una vuelo. Posee la lógica para calcular el retraso que tuvo y el del vuelo asociado si existiese.
\end{itemize}


\section{Resultados}

No fuimos capaces de ejecutar nuestros experimentos con VisualJCLEC. Después de configurar todo correctamente (o al menos cómo lo hicimos en la prática anterior), al ejecutar el experimento obteníamos siempre un \lstinline|NullPointerException| proveniente de su clase \lstinline|RunExperiment|. Tras perder unas cuantas horas intentado localizar la fuente del error, no fuimos capaces de encontralo.

Las dos versiones de la práctica se encuentran implementadas y funcionan correctamente si se ejecutan sin VisualJCLEC. 

El tiempo perdido con VisualJCLEC unido a las malas fechas en las que nos encontramos, nos impidieron analizar los resultados obtenidos y completar así la practica.

\section{Análisis}

Desafortunadamente no se tuvo tiempo para analizar los resultados.
	 
\section{Conclusiones}

Sin analizar los resultados no podemos sacar conclusiones de las implementaciones realizadas. 

% -------------------------------------------------------------
% Bibliography
% -------------------------------------------------------------
\newpage
\bibliography{citations}
\bibliographystyle{plainnat}

\end{document}
